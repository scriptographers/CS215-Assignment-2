\documentclass[11pt, fleqn]{article}

\usepackage{amsmath}
\usepackage{amssymb}
\usepackage{amsthm}
\usepackage{hyperref}
\usepackage{ulem}
\usepackage{enumitem}
\usepackage[left=0.75in, right=0.75in, bottom=0.75in]{geometry}
\usepackage{graphicx}
\usepackage{float}

\usepackage{sectsty}
\sectionfont{\centering}

\usepackage[perpage]{footmisc}

\usepackage{fancyhdr}
\pagestyle{fancy}
\fancyhf{}
\lhead{190100044 \& 190100055}
\rhead{CS 215: Assignment 2}
\renewcommand{\footrulewidth}{1.0pt}
\cfoot{Page \thepage}

\setlength{\parindent}{0em}
\renewcommand{\arraystretch}{2}%

\title{Assignment 2: CS 215}
\author{
\begin{tabular}{|c|c|}
     \hline
     Devansh Jain & Harshit Varma \\
     \hline
     190100044 & 190100055 \\
     \hline
\end{tabular}
}
\date{September 6, 2020}

\begin{document}

\maketitle
\tableofcontents
\thispagestyle{empty}
\setcounter{page}{0}


\newpage
\section*{Question 1}
\addcontentsline{toc}{section}{Question 1}
\setcounter{equation}{0}

Given that $X_1, \dots, X_n$ are $n$ Independent Identically distributed random variables with cdf $F_X(x)$ and pdf $f_X(x) = F_X'(x)$.\\

To determine cdf and pdf for $Y_1 = \text{max}(X_1, \dots, X_n)$
\begin{equation*}
\begin{split}
    F_{Y_1}(x) &= P(Y_1 \le x) \\
        &= \prod_{i=1}^{n} P(X_i \le x) \hspace{3em} (\text{As $Y_1$ = max($X_i$s)}) \\
        &= P(X \le x)^n \hspace{3em} (\text{$X_i$s are identically distributed}) \\
    F_{Y_1}(x) &= F_X(x)^n \\
    f_{Y_1}(x) &= n \cdot F_X(x)^{n-1} \cdot f_X(x) \hspace{3em} (\text{Differentiation by $x$}) \\
\end{split}
\end{equation*}

To determine cdf and pdf for $Y_2 = \text{min}(X_1, \dots, X_n)$
\begin{equation*}
\begin{split}
    F_{Y_2}(x) &= 1 - P(Y_2 \ge x) \\
        &= 1 - \prod_{i=1}^{n} P(X_i \ge x) \hspace{3em} (\text{As $Y_1$ = min($X_i$s)}) \\
        &= 1 - (P(X \ge x))^n \hspace{3em} (\text{$X_i$s are identically distributed}) \\
        &= 1 - (1 - P(X \le x))^n \\
    F_{Y_2}(x) &= 1 - (1 - F_X(x))^n \\
    f_{Y_2}(x) &= n \cdot (1 - F_X(x))^{n-1} \cdot f_X(x) \hspace{3em} (\text{Differentiation by $x$}) \\
\end{split}
\end{equation*}

The cdf and pdf of $Y_1 = \text{max}(X_1, \dots, X_n)$ are $F_X(x)^n$ and $n \cdot F_X(x)^{n-1} \cdot f_X(x)$ respectively. \\
The cdf and pdf of $Y_2 = \text{min}(X_1, \dots, X_n)$ are $1 - (1 - F_X(x))^n$ and $n \cdot (1 - F_X(x))^{n-1} \cdot f_X(x)$ respectively.







\newpage
\section*{Question 2}
\addcontentsline{toc}{section}{Question 2}
\setcounter{equation}{0}

\subsection*{Part 1}
It's given that $X$ belongs to a Gaussian Mixture Model.\\
$$ f_X(x) = \sum_{i=1}^{k}p_i \mathcal{N}(\mu_i, \sigma_i^2) $$
Throughout the rest of this question, we shall use $G_i$ for denoting $\mathcal{N}(\mu_i, \sigma_i^2) $.
\subsubsection*{Calculating E($X$):}
We know that $E(X) = \int_{-\infty}^{\infty} x G_i dx = \mu_i$ (Expection of a Gaussian)\\
Thus,
$$
\begin{aligned}
    E(X) &= \int_{-\infty}^{\infty}x f_X(x) dx\\
    &= \int_{-\infty}^{\infty} x \sum_{i=1}^{k}p_i G_i dx\\
    &= \sum_{i=1}^{k}p_i \int_{-\infty}^{\infty} x G_i dx\\
    &= \boxed{ \sum_{i=1}^{k}p_i \mu_i }
\end{aligned}
$$
\subsubsection*{Caluclating Var($X$):}
We know that:
$$ 
Var(X) = E((X-\mu)^2) = E(X^2) - E(X)^2 = E(X^2) - \mu^2 $$
and,
$$
\begin{aligned}
Var(X) &= \sigma_i^2 = E(X^2)-\mu_i^2 \text{ for a Gaussian } G_i \\
\text{Thus, } E(X^2) &= \int_{-\infty}^{\infty} x^2 G_i dx = \sigma_i^2+\mu_i^2
\end{aligned}
$$
We shall now calculate $E(X^2)$
$$
\begin{aligned}
    E(X^2) &= \int_{-\infty}^{\infty} x^2 \sum_{i=1}^{k}p_i G_i dx\\
    &= \sum_{i=1}^{k}p_i \int_{-\infty}^{\infty} x^2 G_i dx\\
    &= \sum_{i=1}^{k}p_i (\sigma_i^2 + \mu_i^2)
\end{aligned}
$$
Thus, now we have:
$$
Var(X) = E(X^2)-\mu^2 = \boxed{ \sum_{i=1}^{k}p_i (\sigma_i^2 + \mu_i^2) - (\sum_{i=1}^{k}p_i \mu_i)^2 }
$$
\\
\subsubsection*{Caluclating the MGF:}
By the definition of the MGF, $\phi_X (t) = E(e^{tX})$\\
We also know that for a gaussian $G_i$, we have $\phi_X (t) = \int_{-\infty}^{\infty}e^{tx} G_i dx = \exp(\mu_i t + \frac{(\sigma_i t)^2}{2})$\\
Thus for the given $X$,
$$
\begin{aligned}
    \phi_X (t) &= E(e^{tX}) \\
    &= \int_{-\infty}^{\infty}e^{tX} \sum_{i=1}^{k}p_i G_i dx\\
    &= \sum_{i=1}^{k}p_i \int_{-\infty}^{\infty}e^{tX} G_i dx\\
    &= \boxed{\sum_{i=1}^{k}p_i\exp(\mu_i t + \frac{(\sigma_i t)^2}{2})}
\end{aligned}
$$
\subsection*{Part II}
It is given that $Z = \sum_{i=1}^{k}p_iX_i $ where $X_i \thicksim G_i$ are independent random variables.\\
By the properties of Gaussian distribution $G_i$ we know that:\\
$$
\begin{aligned}
E(X_i) &= \mu_i\\
Var(X_i) &= \sigma_i^2\\
\end{aligned}
$$
\subsubsection*{Calculating E($Z$):}
$$
\begin{aligned}
    E(Z) &= E( \sum_{i=1}^{k}p_iX_i)\\
    &= \sum_{i=1}^{k}p_i E(X_i) \hspace{3em} (\text{Linearity of Expectation})\\
    &= \boxed{ \sum_{i=1}^{k}p_i\mu_i }
\end{aligned}
$$
\subsection*{Calculating Var($Z$)}
$$
\begin{aligned}
    Var(Z) &= Var( \sum_{i=1}^{k}p_iX_i)\\
    &= \sum_{i=1}^{k}p_i^2 Var(X_i) \hspace{3em} (\text{as $\{ X_i\}_{i=1}^k$ are independent and $Var(aX) = a^2Var(X)$})\\
    &= \boxed{ \sum_{i=1}^{k}p_i^2\sigma_i^2 }
\end{aligned}
$$

\subsection*{Calculating the MGF}
We know that for a Gaussian $X_i \thicksim G_i$, we have:
$$ \phi_{X_{i}} (t) = \int_{-\infty}^{\infty}e^{tx} G_i dx = \exp(\mu_i t + \frac{(\sigma_i t)^2}{2}) $$
We also know the following properties of $\phi_{X}(t)$:
$$ \phi_{(aX)}(t) = \phi_{X}(at) $$
$$ \phi_{X+Y}(t) =  \phi_{X}(t) \phi_{Y}(t) \text{ for independent X, Y }$$
Thus, using the above 2 properties, and the fact that $\{ X_i\}_{i=1}^k$ are independent,
$$
\begin{aligned}
    \phi_{Z}(t) &= \prod_{i=1}^{k}\phi_{(p_iX_i)}(t)\\
    &= \prod_{i=1}^{k}\phi_{X_i}(p_it)\\
    &= \prod_{i=1}^{k}\exp(\mu_i p_i t + \frac{(\sigma_i p_i t)^2}{2}) \hspace{1em} (\text{as } X_i \thicksim G_i ) \\
    &= \boxed{\exp(t\sum_{i=1}^k\mu_i p_i + \frac{t^2}{2}\sum_{i=1}^{k}p_i^2\sigma_i^2} \hspace{1em} (\text{as } \exp(a)\exp(b) = \exp(a+b)) 
\end{aligned}
$$
\subsection*{Calculating the PDF}
The obtained MGF of $Z$ is same as that of a gaussian of mean $\mu = \sum_{i=1}^k(\mu_i p_i)$ and variance $\sigma^2 = \sum_{i=1}^{k}p_i^2\sigma_i^2$\\
Thus, $Z \thicksim \mathcal{N}(\mu, \sigma^2)$. (Using the Moment Generating Function (MGF) Uniqueness Theorem\footnote{For a given random variable, the MGF and PMF \textbf{uniquely} determine each other.})\\
Therefore, the PDF of Z will be:
$$ \boxed{f_Z(z) = \frac{1}{\sigma \sqrt{2\pi}} \exp\bigg(-\frac{(z-\mu)^2}{2\sigma^2}\bigg)} $$


\newpage
\section*{Question 3}
\addcontentsline{toc}{section}{Question 3}
\setcounter{equation}{0}
\textbf{To prove:}
$$ 
P(X-\mu \ge \tau) \le \frac{\sigma^2}{\sigma^2+\tau^2} \hspace{2em} \text{for } \tau > 0
$$
$$ 
P(X-\mu \ge \tau) \ge 1 - \frac{\sigma^2}{\sigma^2+\tau^2} \hspace{2em} \text{for } \tau < 0
$$
We shall prove some lemmas before coming to the actual proof:\\
(Consider $X$ to be a random variable and $a \ge 0$.)
\begin{enumerate}
    \item $P(X \ge a) = P(X+b \ge a+b)$\\
    \label{lemma1}
    Note that $ (X \ge a) \iff (X+b \ge a+b) $,
    thus this directly leads to the above statement.
    \item $ P(X^2 \ge a^2) \ge P(X \ge a) $\\
    \label{lemma2}
    This is because of the fact that:
    $$ X^2 \ge a^2 \Rightarrow |X| \ge a \Rightarrow (X > a) \lor (X < -a) $$
    Thus, $ P(X^2 \ge a^2) = P(X > a) + P(X < -a)$. \\
    As probabilities are always  $\ge 0$, $ P(X^2 \ge a^2) \ge P(X > a) $
\end{enumerate}
Markov's Inequality states that, for $a > 0$ and a random variable $X$ which always takes positive values, we have:
$$ P(X \ge a) \le \frac{E(X)}{a} $$
Consider a random variable $Y := X - \mu$ and $Z := Y + b$ and consider the case where $\tau > 0$. \\
Also consider $b \ge 0$, thus we have $(\tau + b) \ge 0$\\
Thus, as $Z^2 \ge 0$ and $ (\tau + b)^2 > 0 $, we can apply the Markov's inequality these:
$$ P((Y+b)^2 \ge (\tau + b)^2) \le \frac{E((Y+b)^2)}{(\tau + b)^2} $$
Also, by Lemma \ref{lemma2}, we have $ P((Y+b)^2 \ge (\tau + b)^2) \ge P(Y+b \ge \tau + b) $.\\
By Lemma \ref{lemma1}, $ P((Y+b) \ge (\tau + b)) = P(Y \ge \tau) $\\
Now, using the linearity of the expectation operator,
$$
\begin{aligned}
    E((Y+b)^2) &= E(Y^2) + E(2Yb) + E(b^2)\\
    &= E(Y^2) + 2bE(Y) + b^2 \hspace{2em} \text{(expectation of a constant is the constant itself)}\\
    &= \sigma^2 + 0 + b^2 = \sigma^2 + b^2 \hspace{2em} (E((X-\mu)^2) = \sigma^2 \text{ and $ E(X-\mu) = 0 $)}
\end{aligned}
$$
Thus we have,
$$
    P(Y \ge \tau) \le \frac{\sigma^2 + b^2}{(\tau + b)^2}
$$
To further tighten the inequality, we need to minimize the function $ g(b) =  \frac{\sigma^2 + b^2}{(\tau + b)^2} $, solving $ g'(b) = 0 $ will yield us an extrema $b_0$:
$$ g'(b) = \frac{2b}{(\tau + b)^2} - \frac{2(\sigma^2 + b^2)}{(\tau + b)^3} $$
Equating it to zero, we get $ b_0 = \frac{\sigma^2}{\tau} $, furthermore as $g''(b) > 0$, this indeed is the minima.\\
Thus, after putting $Y = X - \mu$ and $b = b_0$, we get:
$$
    \boxed{P((X-\mu) \ge \tau) \le \frac{\sigma^2}{\sigma^2+\tau^2}} \hspace{2em} (\text{for } \tau > 0)
$$
Now, as $(X-\mu)^2 = (\mu - X)^2$, the inequality $ P((Y) \ge \tau) \le \frac{\sigma^2}{\sigma^2+\tau^2} $ is valid even for $Y = \mu -X$.\\
Thus for $ \tau < 0 $ we can still use the above result for $ -\tau $.\\
$$
\begin{aligned}
P((\mu - X) \ge (-\tau)) &\le \frac{\sigma^2}{\sigma^2+\tau^2}\\
P((X-\mu) < \tau) &\le \frac{\sigma^2}{\sigma^2+\tau^2}\\
1-P((X-\mu) \ge \tau) &\le \frac{\sigma^2}{\sigma^2+\tau^2} \hspace{1em} (P(X < a) = 1 - P(X \ge a))\\
\end{aligned}
$$
$$
\boxed{P((X-\mu) \ge \tau) \ge 1 - \frac{\sigma^2}{\sigma^2+\tau^2}} \hspace{1em} (\text{for } \tau < 0)
$$









\newpage
\section*{Question 4}
\addcontentsline{toc}{section}{Question 4}
\setcounter{equation}{0}

By definition, MGF of a random variable X for parameter t is $\phi_X(t) = \int_{-\infty}^{\infty} e^{tu} f_X(u) du$.\\

\begin{equation}
    \label{1}
    \begin{split}
        e^{-tx} \phi_X(t) &= \int_{-\infty}^{\infty} e^{t(u-x)} f_X(u) du \\
            &\ge \int_{-\infty}^{\infty} (1 + t(u-x)) f_X(u) du \hspace{3em} (e^x \ge (1 + x)) \\
    \end{split}
\end{equation}
\\
For $t \ge 0$,
\begin{equation}
    \begin{split}
        e^{-tx} \phi_X(t) &\ge \int_{x}^{\infty} (1 + t(u-x)) f_X(u) du \hspace{3em} (\text{From \ref{1}}) \\
            &\ge [\int_{x}^{\infty} f_X(u) du] + [t \cdot \int_{x}^{\infty} (u-x) f_X(u) du] \\
            &\ge P(X \ge x) + [t \cdot \int_{x}^{\infty} (u-x) f_X(u) du]
    \end{split}
\end{equation}
As $\quad t \ge 0;\quad u \ge x; \quad f_X(u) \ge 0 \  \forall u\in(-\infty, \infty), \quad$ we get
\begin{equation}
    \label{3}
    P(X \ge x) \le e^{-tx} \phi_X(t) \quad \forall t \ge 0
\end{equation}
\\
For $t \le 0$,
\begin{equation}
    \begin{split}
        e^{-tx} \phi_X(t) &\ge \int_{-\infty}^{x} (1 + t(u-x)) f_X(u) du \hspace{3em} (\text{From \ref{1}}) \\
            &\ge [\int_{-\infty}^{x} f_X(u) du] + [t \cdot \int_{-\infty}^{x} (u-x) f_X(u) du] \\
            &\ge P(X \le x) + [t \cdot \int_{-\infty}^{x} (u-x) f_X(u) du]
    \end{split}
\end{equation}
As $\quad t \le 0;\quad u \le x; \quad f_X(u) \ge 0 \  \forall u\in(-\infty, \infty), \quad$ we get
\begin{equation}
    P(X \le x) \le e^{-tx} \phi_X(t) \quad \forall t \ge 0
\end{equation}

\newpage
Now, $X = \sum_{i=1}^{n} X_i$, where $X_i$ are independent Bernoulli random variables with mean $p_i$.\\
$E(X) = \mu = \sum_{i=1}^{n} p_i$.\\
\begin{equation}
    \label{6}
    \begin{split}
        \phi_X(t) &= \prod_{i=1}^{n} \phi_{X_i}(t) \hspace{3em} (X_i \text{s are independent random variables}) \\
            &= \prod_{i=1}^{n} (1 - p_i + p_i e^{t}) \hspace{3em} (X_i \text{s are Bernoulli random variables}) \\
            &\le \prod_{i=1}^{n} (e^{p_i(e^t - 1)}) \hspace{3em} (e^x \ge (1 + x)) \\
            &\le exp(\sum_{i=1}^{n} p_i(e^t - 1)) \\
            &\le exp(\mu (e^t - 1)) \\
    \end{split}
\end{equation}
\begin{equation}
    \begin{split}
        P(X > (1 + \delta) \mu) &\le e^{-t \cdot ((1 + \delta) \mu)} \phi_X(t) \quad \forall t \ge 0 \hspace{3em} (\text{From \ref{3}}) \\
            &\le exp((-(1 + \delta) t \mu)) + \mu (e^t - 1)) \hspace{3em} (\text{From \ref{6}})
    \end{split}
\end{equation}
Therefore,
\begin{equation}
    \label{8}
    P(X > (1 + \delta) \mu) \le \frac{e^{\mu(e^t-1)}}{e^{(1+\delta)t\mu}} \quad \forall t \ge 0, \ \delta \ge 0
\end{equation}
To get a tighter bound, we need the minimum attainable value of the RHS of Eqn \ref{8}. \\
As $e^x$ is a monotonically increasing function, we need to minimize $(-(1 + \delta) t \mu)) + \mu (e^t - 1)$.\\
$\frac{d}{dt} ((-(1 + \delta) t \mu)) + \mu (e^t - 1)) = (\mu e^t - (1 + \delta) \mu) = 0 \quad \implies \quad t = ln(1 + \delta)$ \\
\begin{equation}
    P(X > (1 + \delta) \mu) \le \frac{e^{\mu\delta}}{e^{\mu(1+\delta)(ln(1 + \delta))}} \quad \forall \delta \ge 0 \\
\end{equation}










\newpage
\section*{Question 5}
\addcontentsline{toc}{section}{Question 5}
\setcounter{equation}{0}











\newpage
\section*{Question 6}
\addcontentsline{toc}{section}{Question 6}
\setcounter{equation}{0}











\newpage
\section*{Question 7}
\addcontentsline{toc}{section}{Question 7}
\setcounter{equation}{0}

MGF of Multinomial distribution of $\mathbf{X} = (X_1, \dots, X_k)$, where $\sum_{i=1}^{k} X_i = n$ is $\phi_{\mathbf{X}}(\mathbf{t}) = (p_1 e^{t_1} + \cdots + p_k e^{t_k})^{n}$, where $\mathbf{t} = (t_1, \dots, t_k)$ and $p_i$ represent probability of $X_i$. \\
\begin{equation*}
    \begin{split}
        \frac{\partial}{\partial t_i} \phi_{\mathbf{X}}(\mathbf{t}) &= n \cdot (p_1 e^{t_1} + \cdots + p_k e^{t_k})^{n-1} \cdot (p_i e^{t_i}) \\
        \mu_i = E(X_i) &= \frac{\partial}{\partial t_i} \phi_{\mathbf{X}}(\mathbf{t}) \bigg\rvert_{\mathbf{t} = \mathbf{0} = (0, \dots, 0)} \\
            &= n \cdot (p_1 + \cdots + p_k)^{n-1} \cdot p_i \\
            &= n \cdot p_i \hspace{3em} (\sum_{n=1}^{k} p_i = 1) 
    \end{split}
\end{equation*}

\begin{equation*}
    \begin{split}
        \frac{\partial^2}{\partial t_i^2} \phi_{\mathbf{X}}(\mathbf{t}) &= n(n-1) \cdot (p_1 e^{t_1} + \cdots + p_k e^{t_k})^{n-2} \cdot (p_i e^{t_i}) + n \cdot (p_1 e^{t_1} + \cdots + p_k e^{t_k})^{n-1} \cdot (p_i e^{t_i}) \\
        E(X_i^2) &= \frac{\partial^2}{\partial t_i^2} \phi_{\mathbf{X}}(\mathbf{t}) \bigg\rvert_{\mathbf{t} = \mathbf{0} = (0, \dots, 0)} \\
            &= n(n-1) \cdot (p_1 + \cdots + p_k)^{n-2} \cdot p_i + n \cdot (p_1 + \cdots + p_k)^{n-1} \cdot p_i \\
            &= n(n-1) \cdot p_i + n \cdot p_i \hspace{3em} (\sum_{n=1}^{k} p_i = 1) \\
            &= n^2 \cdot p_i \\
        Cov(X_i, X_i) &= Var(X_i) \\
            &= E(X_i^2) - E(X_i)^2 \\
            &= n^2 \cdot p_i - (n \cdot p_i)^2 \\
            &= n^2 \cdot p_i (1 - p_i) \\
    \end{split}
\end{equation*}

For $i \ne j$,
\begin{equation*}
    \begin{split}
        \frac{\partial^2}{\partial t_j \partial t_i} \phi_{\mathbf{X}}(\mathbf{t}) &= n(n-1) \cdot (p_1 e^{t_1} + \cdots + p_k e^{t_k})^{n-2} \cdot (p_i e^{t_i}) \cdot (p_j e^{t_j}) \\
        E(X_i \cdot X_j) &= \frac{\partial^2}{\partial t_j \partial t_i} \phi_{\mathbf{X}}(\mathbf{t}) \bigg\rvert_{\mathbf{t} = \mathbf{0} = (0, \dots, 0)} \\
            &= n(n-1) \cdot (p_1 + \cdots + p_k)^{n-2} \cdot p_i \cdot p_j \\
            &= n(n-1) \cdot p_i  \cdot p_j \hspace{3em} (\sum_{n=1}^{k} p_i = 1) \\
        Cov(X_i, X_j) &= E[(X_i - \mu_i)(X_j - \mu_j)] \\
            &= E[(X_i - E(X_i))(X_j - E(X_j))] \\
            &= E(X_i \cdot X_j) - E(X_i)E(X_j) \\
            &= n(n-1) \cdot p_i  \cdot p_j - (n \cdot p_i)(n \cdot p_j) \\
            &= (-n) \cdot p_i \cdot p_j  \\
    \end{split}
\end{equation*}

The co-variance matrix $\mathbf{C}$ is given by:
\begin{equation*}
    \begin{split}
        C_{ii} &= n \cdot p_i (1 - p_i) \\
        C_{ij} &= (-n) \cdot p_i \cdot p_j \quad \forall \ i \ne j \\
    \end{split}
\end{equation*}









\end{document}
